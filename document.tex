\documentclass[11pt,a4paper]{book}
\usepackage[latin1]{inputenc}
\usepackage{amsmath}
\usepackage{amsfonts}
\usepackage{amssymb}
\usepackage{graphicx}
\author{Himanshu}
\title{QFT notes}



\begin{document}
\maketitle
\tableofcontents
	
    
\chapter{First Chapter}
\section{Introduction}
	
	
\subsection{This is just for testing}
	
Some text to check whether the things are working or not.
	
\chapter{Basics}
\section{Basics}
This is some text to get the index working.
\glossary{gwrge}

$\mu = \nu$
	
$	i\hbar {\frac {\partial }{\partial t}}\Psi (\mathbf {r} ,t)=\left[{\frac {-\hbar ^{2}}{2\mu }}\nabla ^{2}+V(\mathbf {r} ,t)\right]\Psi (\mathbf {r} ,t)$

$E = \frac { 1 } { 2 } N k _ { \mathrm { B } } T$

$F = m \frac { 2 \pi c k _ { \mathrm { B } } T } { \hbar } = m \frac { 4 \pi c } { \hbar } \frac { E } { N } = m \frac { 4 \pi c ^ { 3 } } { \hbar } \frac { M } { N } = m 4 \pi \frac { G M } { A } = G \frac { m M } { r ^ { 2 } }$

$12 \cdot \frac { - i \lambda } { 4 ! } \int d ^ { 4 } z D _ { F } ( x - z ) D _ { F } ( y - z ) D _ { F } ( z - z )$


$\mathcal { L } = \overline { \psi } \left( i \gamma ^ { \mu } \partial _ { \mu } - m \right) \psi - \frac { 1 } { 4 } F _ { \mu \nu } F ^ { \mu \nu } - e \overline { \psi } \gamma ^ { \mu } \psi A _ { \mu }$


\begin{equation}
    \mathcal { L } = \overline { \psi } \left( i \gamma ^ { \mu } \partial _ { \mu } - m \right) \psi - \frac { 1 } { 4 } F _ { \mu \nu } F ^ { \mu \nu } - e \overline { \psi } \gamma ^ { \mu } \psi A _ { \mu }
\end{equation}


\begin{equation}
    \frac { \left\langle 0 _ { + } | \overline { \psi } ( x ) | 0 _ { - } \right\rangle } { \left\langle 0 _ { + } | 0 _ { - } \right\rangle } = \mathrm { i } \int \left( \mathrm { d } x ^ { \prime } \right) \int \frac { \mathrm { d } ^ { 3 } \mathbf { p } } { ( 2 \pi ) ^ { 3 } 2 p ^ { 0 } } \mathrm { e } ^ { - \mathrm { i } p \left( x ^ { \prime } - x \right) } \overline { \eta } \left( x ^ { \prime } \right) ( \gamma p + m )
\end{equation}

\begin{equation}
    \left. \begin{array} { l } { \mathrm { i } \mathrm { W } _ { 21 } = \int \sum _ { \sigma } \frac { m } { p ^ { 0 } } \frac { \mathrm { d } ^ { 3 } \mathbf { p } } { ( 2 \pi ) ^ { 3 } } \left[ \mathrm { i } \overline { \eta } _ { 2 } ( p ) u ( \mathbf { p } , \sigma ) \right] \left[ \mathrm { i } \overline { u } ( \mathbf { p } , \sigma ) \eta _ { 1 } ( p ) \right] } \\ { + \int \sum _ { \sigma } \frac { m } { p ^ { 0 } } \frac { \mathrm { d } ^ { 3 } \mathbf { p } } { ( 2 \pi ) ^ { 3 } } \left[ - \mathrm { i } \overline { v } ( \mathbf { p } , \sigma ) \eta _ { 2 } ( - p ) \right] \left[ - \mathrm { i } \overline { \eta } _ { 1 } ( - p ) v ( \mathbf { p } , \sigma ) \right] } \end{array} \right.
\end{equation}

\begin{equation}
    \left.\begin{aligned} \partial _ { r } & \left[ \sqrt { - g _ { 2 } } \left\{ \left( \rho h + b ^ { 2 } \right) u ^ { r } u _ { \alpha } - b ^ { r } b _ { \alpha } + \delta _ { \alpha } ^ { r } \left( P + \frac { b ^ { 2 } } { 2 } \right) \right\} \right] \\& - \frac { 1 } { 2 } \sqrt { - g _ { 2 } } \left[ \left( \rho h + b ^ { 2 } \right) u ^ { \mu } u ^ { \nu } - b ^ { \mu } b ^ { \nu } + \left( P + \frac { b ^ { 2 } } { 2 } \right) g ^ { \mu \nu } \right] \partial _ { \alpha } g _ { \mu \nu } = 0 \end{aligned} \right.
\end{equation}


\begin{equation}
    \left.\begin{aligned} M _ { \mathrm { K } } & = - \frac { 1 } { 4 \pi } \oint _ { \mathcal { H } } d S _ { a } n _ { b } \nabla ^ { a } \xi ^ { b } - \frac { 1 } { 4 \pi } \int _ { \Sigma _ { t } ^ { \prime } } d V n _ { b } \nabla _ { a } \nabla ^ { a } \xi ^ { b } \\ & = - \frac { 1 } { 4 \pi } \oint _ { \mathcal { H } } d S _ { a } n _ { b } \nabla ^ { a } \xi ^ { b } + \frac { 1 } { 4 \pi } \int _ { \Sigma _ { t } ^ { \prime } } d V n _ { b } R _ { a } ^ { b } \xi ^ { a } \\ & = - \frac { 1 } { 4 \pi } \oint _ { \mathcal { H } } d S _ { a } n _ { b } \nabla ^ { a } \xi ^ { b } + 2 \int _ { \Sigma _ { t } ^ { \prime } } d V \left( T _ { a b } - \frac { 1 } { 2 } g _ { a b } T \right) \xi ^ { a } n ^ { b } \end{aligned} \right.
\end{equation}
	
	
\begin{equation}
\left.\begin{aligned} d s ^ { 2 } = & - d \overline { t } ^ { 2 } + d x ^ { 2 } + d y ^ { 2 } + d z ^ { 2 } \\ & + \frac { 2 M r ^ { 3 } } { r ^ { 4 } + a ^ { 2 } z ^ { 2 } } \left( \frac { r ( x d x + y d y ) - a ( x d y - y d x ) } { r ^ { 2 } + a ^ { 2 } } + \frac { z d z } { r } + d \overline { t } \right) ^ { 2 } \end{aligned} \right.    
\end{equation}


\begin{equation}
    \left. \begin{array} { c } { W ^ { ( \mathrm { e } ) } = \int ( \mathrm { d } x ) \frac { \left( E ^ { 2 } - B ^ { 2 } \right) } { 2 } - [ \frac { 1 } { 8 \pi ^ { 2 } } \int ( \mathrm { d } x ) \int _ { 0 } ^ { \infty } \frac { \mathrm { d } s } { s ^ { 3 } } \mathrm { e } ^ { - s m ^ { 2 } } } \\ { \times \left\{ ( s | \mathrm { e } E | \cot s | \mathrm { e } E | ) ( s | \mathrm { e } B | \operatorname { coth } s | \mathrm { e } B | ) + \frac { ( s \mathrm { e } ) ^ { 2 } \left( E ^ { 2 } - B ^ { 2 } \right) } { 3 } - 1 \right\} ] } \end{array} \right.
\end{equation}






\input{chapters/Integral_approach}

\input{chapters/QFT}


\end{document}
\makeglossary